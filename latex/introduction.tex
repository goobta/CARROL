\section{Introduction}

Super resolution is a classic problem in computer vision involving upscaling a
low resolution image in order to obtain a clearer or more detailed image which
has seen a wide range of use from raw image enhancement to feature detection
within images\cite{Chen,Dong2016,Ledig}. In general, super resolution has been quite successful
when implemented with a Generative Adversarial Network (GAN)\cite{Chen}. Our goal in
this paper is to apply this type of super resolution algorithm with an added
class-conditional feature to various individual subjects in an image to increase
their visibility.

A problem with current methods of super resolution is that normal GANs and deep
learning algorithms have a lot of trouble handling images with multiple
subjects. The output of these algorithms have the possibility of being entirely
unrecognizable compared to the original image\cite{Goodfellow2017}. 
Another problem with current
super resolution approaches is that they operate on an entire image, the issue
being that the difference in resolution between the foreground and background
doesn’t necessarily change that much. This means that there will still be the
same lack of focus on the background in the new image as there was in the
original. The last problem this paper will address is that common super
resolution algorithms only use the original image’s pixels as input data to run
super resolution. While this method has been shown to work quite well, we
believe it can be improved by adding a class condition.

One of the main decisions we had to make was how to go about building the super
resolution algorithm. Since our focus isn’t so much the super resolution
algorithm itself and there has already been extensive research on the
performance of various algorithms, we did some research to find an algorithm
that is accepted by the majority of the community as the best for our purposes
which is a conditional GAN\cite{Chen}. We believe that in employing an SRCGAN with the
supporting processes described, the product of the complete process will be a
more meaningful image that puts emphasis on the more important parts of the
original image.
